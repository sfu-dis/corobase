% This is "sig-alternate.tex" V1.9 April 2009
% This file should be compiled with V2.4 of "sig-alternate.cls" April 2009
%
% This example file demonstrates the use of the 'sig-alternate.cls'
% V2.4 LaTeX2e document class file. It is for those submitting
% articles to ACM Conference Proceedings WHO DO NOT WISH TO
% STRICTLY ADHERE TO THE SIGS (PUBS-BOARD-ENDORSED) STYLE.
% The 'sig-alternate.cls' file will produce a similar-looking,
% albeit, 'tighter' paper resulting in, invariably, fewer pages.
%
% ----------------------------------------------------------------------------------------------------------------
% This .tex file (and associated .cls V2.4) produces:
%       1) The Permission Statement
%       2) The Conference (location) Info information
%       3) The Copyright Line with ACM data
%       4) NO page numbers
%
% as against the acm_proc_article-sp.cls file which
% DOES NOT produce 1) thru' 3) above.
%
% Using 'sig-alternate.cls' you have control, however, from within
% the source .tex file, over both the CopyrightYear
% (defaulted to 200X) and the ACM Copyright Data
% (defaulted to X-XXXXX-XX-X/XX/XX).
% e.g.
% \CopyrightYear{2007} will cause 2007 to appear in the copyright line.
% \crdata{0-12345-67-8/90/12} will cause 0-12345-67-8/90/12 to appear in the copyright line.
%
% ---------------------------------------------------------------------------------------------------------------
% This .tex source is an example which *does* use
% the .bib file (from which the .bbl file % is produced).
% REMEMBER HOWEVER: After having produced the .bbl file,
% and prior to final submission, you *NEED* to 'insert'
% your .bbl file into your source .tex file so as to provide
% ONE 'self-contained' source file.
%

%
% For tracking purposes - this is V1.9 - April 2009

%%\documentclass{sig-alternate}
%%\documentclass[letter]{sig-alternate}
\documentclass{vldb}

\usepackage{microtype}

\usepackage{ifthen} % \isempty macro..
\usepackage{graphicx}
\usepackage[usenames,dvipsnames]{color}
%\usepackage[numbers,sort]{natbib}
\usepackage{verbatim}
%\newcommand{\newblock}{} 
% allow code listings with line numbering
\usepackage{listings}
%define how inlined source code should be displayed
\lstset {
language=C,
basicstyle=\scriptsize,
keywordstyle=\ttfamily\bfseries\color{blue},
commentstyle=\color{OliveGreen},
numbers=none,
numberstyle=\scriptsize,
numbersep=5pt,
tabsize=4,
gobble=4,
xleftmargin=20pt,
escapeinside={(@*}{*@)},
morekeywords={}
}

\newcounter{mycounter}

% by default footnotes are indented on the first line, which makes
% them look pretty ragged. Change them to be flush left with hanging
% indent (similar to references in the bibliography). NOTE: must
% include before hyperref package
\usepackage[hang,splitrule]{footmisc}
\setlength{\footnotemargin}{1em}

\def\thepapertitle{ERMIA: An architecture for fast and robust memory-optimized OLTP}
\def\thepaperkeywords{Indirection array, Epoch-based, OLTP, Concurrency control}
  
% must import before hyperref (and algorithm, which imports it, must come after)
\usepackage{float} 
\usepackage[
  pageanchor=true,
  plainpages=false,
  pdfpagelabels,
  bookmarks,
  bookmarksnumbered,
  pdfborder=0 0 0,  %removes outlines around hyper links in online display
  colorlinks=true,
  linkcolor=blue,
  citecolor=Blue, % OliveGreen
  pdfpagelayout=TwoPageRight,
  pdftitle={\thepapertitle},
  pdfauthor={Ippokratis Pandis},
  pdfsubject={},
  pdfkeywords={\thepaperkeywords},
]{hyperref}

\floatstyle{ruled}
\usepackage{algorithm}
\newfloat{algorithm}{t}{lop}

% make floats waste way less space
% http://dcwww.camd.dtu.dk/~schiotz/comp/LatexTips/LatexTips.html
\renewcommand{\topfraction}{0.85}
\renewcommand{\textfraction}{0.1}
\renewcommand{\floatpagefraction}{0.80}

\setlength{\fboxsep}{1pt}
\let\oldcite\cite
\renewcommand{\cite}[1]{\colorbox{linkbg}{\oldcite{#1}}}
\definecolor{linkbg}{gray}{0.97}
\newcommand{\myref}[2]{\colorbox{linkbg}{\hyperref[#2]{%
      \ifthenelse{\equal{#1}{}}{}{#1 }\ref{#2}}}}

% make a really loud marker for citations I know I don't have yet
%% usage: \citeme
\makeatletter
\newcommand{\citeme}{\@ifstar
  \citemeStar%
  \citemeNoStar%
}
\makeatother

\newcommand{\citemeStar}{\citemeNoStar{citation needed}}
\newcommand{\citemeNoStar}[1]{
  {\color{Magenta}{\bf \fbox{citeme} #1}}
  % missing comment on line above is intentional
}

% TikZ ist kein Zeichenprogramm
\usepackage{tikz}
\usetikzlibrary{plotmarks}
\usetikzlibrary{arrows}
\usetikzlibrary{snakes}
\usetikzlibrary{shapes}
\usetikzlibrary{patterns}

% nice table formatting
\usepackage{booktabs}

% nice fractions, e.g., 1/4 and so forth...
\usepackage{nicefrac}

% flexible but compact enumerations
\usepackage{paralist}

\usepackage{subfigure}
\usepackage[noend]{algorithmic}
\usepackage{soul}
\usepackage{multirow}

%
% please place your own definitions here and don't use \def but
% \newcommand{}{}

\newcommand{\ippo}[1]{\noindent{\color{Red} {\bf \fbox{IP}} {\hl{\it#1}}}}
\newcommand{\ryan}[1]{\noindent{{\bf \fbox{Ryan}} {\hl{\it#1}}}}
\newcommand{\kk}[1]{\noindent{{\bf \fbox{KK}} {\hl{\it#1}}}}
\newcommand{\tianzheng}[1]{\noindent{{\bf \fbox{Tianzheng}} {\hl{\it#1}}}}

%% 1-column
% \labeledfigure{filename}[Short caption]{Long caption}
\newcommand{\labeledfigure}[1]{%
  \def\thelabeledfigure{#1}%
  \labeledfigurerelay%
}
\newcommand{\labeledfigurerelay}[2][]{%
  \begin{figure}[t]%
    \centering%
    \includegraphics{\thelabeledfigure}%
    \ifthenelse{\equal{#1}{}}%
               {\caption{#2}}% then
               {\caption[#1]{#2}}% else
               \label{fig:\thelabeledfigure}%
  \end{figure}%
}

%% 2-column
% \labeledfigure{filename}[Short caption]{Long caption}
\newcommand{\labeledfigurewide}[1]{%
  \def\thelabeledfigurewide{#1}%
  \labeledfigurewiderelay%
}
\newcommand{\labeledfigurewiderelay}[2][]{%
  \begin{figure*}[t]%
    \centering%
    \includegraphics{\thelabeledfigurewide}%
    \ifthenelse{\equal{#1}{}}%
               {\caption{#2}}% then
               {\caption[#1]{#2}}% else
               \label{fig:\thelabeledfigurewide}%
  \end{figure*}%
}

% this bad boy defines a pair of macros for handling chapter-aware
% cross references. If called like this:
% \makereftype{fig}{Fig.}{figure} it would define a macro \figlabel[1]
% (for labeling a figure) and \figref[3][\thechlabel][Fig.]  (for
% referring to it)
\newcommand{\makereftype}[2]{% command stem | default display
  \expandafter\newcommand\csname#1label\endcsname [1]{\label{#1:##1}}
  \expandafter\newcommand\csname#1ref\endcsname [1][]{%
    \expandafter\csname#1refrelay\endcsname%
  }%
  \expandafter\newcommand\csname#1refrelay\endcsname [2][#2]{%
      \myref{##1}{#1:##2}%
  }%
}

\makereftype{fig}{Figure}
\makereftype{sec}{Section}
\makereftype{tbl}{Table}
\makereftype{alg}{Algorithm}
\makereftype{line}{Line}

\makeatletter
\g@addto@macro\algorithm\scriptsize
\makeatother

\newenvironment{code}{\scriptsize}{}

% latex is terrible with widow/orphan lines. This is really
% heavy-handed but nothing else makes it pay attention.
\clubpenalty=10000
\widowpenalty=10000

% IP: Space saving tricks
%\usepackage[small,compact]{titlesec}
%\usepackage[small,it]{caption}
%\usepackage{times}

%% %% IP - HEADER/FOOTER
%% \lhead{}
%% \chead{\MakeUppercase{\bfseries *** please do not distribute ***}}
%% \rhead{}
%% \lfoot{}
%% \cfoot{\large{DRAFT, \today}}
%% \rfoot{}
%% %% IP - EOF HEADER/FOOTER


\begin{document}
\title{\thepapertitle}

\numberofauthors{1}
\author{
\alignauthor
Kangnyeon Kim
~~
Tianzheng Wang
~~
Ryan Johnson
~~
Ippokratis Pandis$^\star$
\bigskip\\
$
\begin{array}{cc}
  \mbox{\affaddr{University of Toronto}} &
  \mbox{$^\star$\affaddr{Cloudera}}\\
  \mbox{{\large \sf {knkim,tzwang,ryan.johnson}@cs.utoronto.ca}} &
  \mbox{{\large \sf ippokratis@cloudera.com}}
\end{array}
$
\\
\tianzheng{(ERMIA: )Fast and robust main-memory databases? feel the title is a bit long}
}

\maketitle

%% Breaking the file into sections

% -*- tex-main-file:"rcu-cc.tex" -*-

\begin{abstract}

Large main memories and massively parallel processors have triggered not only a resurgence of new high-performance transaction processing systems, but also the evolving of heterogeneous read-mostly transactions. However, many of these systems adopt lightweight optimistic concurrency control schemes that are only suitable for short, write-intensive workloads. By analyzing the desired features of main-memory optimized database systems, we argue that it is the system architecture that largely dictates the concurrency control scheme a system can employ, hence the type of workloads that can be gracefully handled.
Therefore, it becomes difficult, if not impossible, to adopt a different concurrency control scheme to robustly handle various workloads without significant changes to the rest of the system. 
%The architecture also largely determines how the physical conflicts between concurrent threads are dealt, and how recovery is achieved.

We believe that transaction processing systems should be designed from the ground up with four basic requirements: to not heavily rely on partitioning; to provide flexible and robust concurrency control for the logical interactions between transactions; to address the physical interactions between threads in a scalable way; and to have a clear recovery methodology.  
We report on the design and implementation a system, called ERMIA, from the ground up to support the requirements we lay out, and show how the resulting architecture achieves these goals without unnecessary sacrifices to performance in other areas. 
%ERMIA's performance is comparable, if not higher, to the performance of the highest-performing open-source in-memory OLTP system in the workloads for which this system was designed for, and it performs up to an order of magnitude higher in workloads with high contention or large read and write sets.

\end{abstract}

%% -*- tex-main-file:"ermia.tex" -*-

\section{Introduction}
\seclabel{intro}

%1. new hw trend has led to new systems
Modern systems with massively parallel processors and large main memories have inspired a new breed of high-performance, memory-optimized transaction processing systems \cite{Kallman+08,PandisJHA10,KemperN11,LarsonBDFPZ11,LevandoskiLSSW15,TuZKLM13}. These systems leverage spacious main memory to fit the whole working set in DRAM with streamlined, memory-friendly data structures. Further, optimizations for multicore and multi-socket hardware allow a much higher level of parallelism compared to conventional database systems. With disk overheads and delays removed, transaction latencies drop precipitously and worker threads can usually execute transactions to completion without interruption. The result is a welcome reduction in contention at the logical level and less pressure on whatever concurrency control (CC) scheme might be in place. A less welcome result is an increasing pressure for scalable data structures and algorithms to cope with the growing number of threads that concurrently execute transactions and need to communicate \cite{JohnsonPA13}.

\labeledfigurewide{fig-write-ratio}{Performance of a memory-efficient OLTP engine with lightweight optimistic concurrency control, as the ratio of writes increases (left); and as the size of the database decreases (right).\kk{remove no-log from the graph, also we don't have detail about this experiment.}}

\vspace{2mm}
{\bf Interactions at the logical level.} 
%2. how current workloads make cc important again.
Many designs exploit the reduction in the pressure on CC, by employing very optimistic and lightweight schemes, boosting even further the performance of these systems on suitable workloads.
But, as is usually the case, it appears that database workloads stand ready to absorb any and all concurrency gains the memory-optimized systems have to offer. In particular, there is high demand for database systems that can readily serve heterogeneous database workloads, blending the gap between transaction and analytical processing. This trend is at least partly enabled by the improved concurrency and reduced contention offered by memory-optimized systems \cite{Farber+12}. Mixed workloads have two significant impacts on CC, however. First, the write/read ratio decreases from 1:2 (e.g. TPC-C) to 1:10 or less (e.g. TPC-E \cite{Chen+10,TozunPKJA13}), usually {\it by increasing the number of reads as the number of writes remains stable}.
Second, workloads frequently include some fraction of large transactions that are {\it read-mostly rather than read-only}---a trend reflected in the TPC-E benchmark. Unfortunately, both of these workload properties result in larger effective concurrency control footprints, putting pressure on the CC scheme. 
Therefore, going forward and as the industry shifts to heterogeneous workloads served by memory-optimized engines, it is vital for them to employ effective and robust CC schemes. 


We observe that the CC schemes currently in vogue with memory-optimized systems are not robust under contention, particularly when short write-intensive transactions coexist with longer \textit{read-mostly} transactions.
The two main families of approaches can be loosely classified as two-phase locking (2PL) \cite{GrayR92} and optimistic concurrency control (OCC) \cite{KungR81}. 2PL is common in traditional disk-oriented systems, and is often criticized because of high overheads, its policy of blocking transactions (leading to deadlocks and other scheduling problems), and a tendency to ``lock up'' (performance crash) once the aggregate transactional footprint grows too large, a state quickly attained when heavy read-mostly transactions enter the system. OCC, on the other hand, never blocks readers---and may not even block writers---thus avoiding most scheduling issues. Although they differ in details, the rising generation of memory-optimized systems almost universally adopts a form of OCC that is effectively single-versioned, with read footprint validation at pre-commit.  Two systems that characteristically employ this type of OCC are Microsoft's Hekaton \cite{LarsonBDFPZ11} and Silo \cite{TuZKLM13}. This type of approach suffers badly in highly concurrent workloads \cite{YuBPDS14} because transactions must abort if any portion of their read footprint is overwritten before they commit. 
In \figref{fig-write-ratio} we demonstrate how the performance of Silo, a representative of the camp of transaction processing engines with lightweight OCC whose code is publicly available, degrades as transactions have larger read footprint or when contention increases. (\secref{eval:setup} has details about the experimental setup.) \figref{fig-write-ratio}(left) shows that it just takes 0.1\% or 1\% of the touched records to be updates for the transaction throughput to drastically drop. While \figref{fig-write-ratio}(right) shows that the abort rate grows quickly as the same number of threads operate on smaller TPC-C databases, thereby on higher contention.


\vspace{2mm}
{\bf Interactions at the physical level.} 
But it is not only the interaction at the logical level that should be central to the design of a memory-optimized transaction processing engine. As commodity server hardware becomes increasingly parallel~\footnote{For example, Intel's new server-grade processor, Haswell-EP, has up to 18 cores (36 hyperthreads) per socket.} many of the low-level issues (latching, thread scheduling, etc..) and design decisions---at the architecture level---need to be revisited. The form of logging used, the storage management architecture, and scheduling policies for worker threads can impose drastic constraints on which forms of CC can be implemented at all, let alone efficiently. 
It can be difficult or impossible to adopt a different CC scheme without significant changes to the rest of the system. 
For example, it was reported in \cite{PortsG12} that the implementation effort required to add support for serializable snapshot isolation (SSI) in Postgres was very high, due to a lack of supporting infrastructure in the system. 
The point is not that such design choices should be avoided, but rather that they should be made only with a full awareness of the consequences for concurrency control. 

\vspace{2mm}
{\bf Physical partitioning.} Some systems sidestep the issues of logical and physical contention entirely---along with the accompanying implementation complexity---by adopting physical partitioning and a single-threaded transaction execution model \cite{Kallman+08,KemperN11}. This execution model introduces a different set problems for mixed workloads and for workloads that are inherently difficult to partition.  Given the developments in scaling-out the performance of distributed OLTP systems, especially for easy-to-partition workloads, e.g. \cite{Corbett+12,BailisFHGS14,ThomsonA10}, as well as for high availability and cost-effectiveness reasons, we predict that the successful architectures will combine scale-out solutions build on top of non-partitioning-based scale-up engines within each node.
Therefore, we focus on the performance of non-partitioning-based memory-optimized engines within a single node.

\vspace{2mm}
{\bf ERMIA.} 
In \secref{desired} we lay out the design principles that we believe are critical for transaction processing engines in the environment of highly-parallel servers with ample main memory. Next, on \secref{design}, we present {\em ERMIA}, a memory-optimized transaction processing architecture that carefully combines several techniques---epoch-based resource management, indirection arrays \cite{SadoghiRCB13}, and a specially-designed log manager---to enable robust CC, scalable thread interactions, and easy recovery.  
\secref{eval} compares the performance of an ERMIA prototype against a representative of the new breed of memory-optimized shared-everything transaction processing systems, and shows how the resulting architecture achieves its goals without necessarily sacrificing performance in other areas.

\section{Design directions}
\seclabel{desired}

In this section we briefly discuss desired properties of transaction processing system architectures. We primarily focus on three areas: the concurrency control mechanism that determines the interaction between concurrent transactions at the logical level; the mechanism that controls the interaction/communication of threads at the physical level; and recovery. As we already argued in \secref{intro}, we are aiming for a scalable single-node design that achieves scale-up with minimal physical partitioning.  

\vspace{2mm} 
{\bf Append-only storage.}
Append-only storage allows drastic simplifications of both I/O patterns and corner cases in the code. Combining a carefully designed log manager and indirection arrays (below) produces a single-copy system where the log is a significant fraction of the database. Records graduate from the log to secondary storage only if they go a long time with no updates. The resulting system is also easier to recover, as both undo and redo are largely unnecessary---the log can be truncated at the first hole without losing any committed work.

\vspace{2mm} 
{\bf Logging.}
System designers should also treat recovery as a first class citizen when designing a transaction processing system. Log managers are a well-known source of complexity and bottlenecks in a database system. Many of the recent proposals do not provide a thorough design for recovery, in stark contrast to more venerable loging protocols such as ARIES \cite{MohanHLPS92}.

The in-memory log buffer is a central point of communication that simplifies a variety of other tasks in the system, but which tends to kill scalability. Past work has attempted to optimize \cite{JohnsonPSAA10} or distribute \cite{WangJ14} the log, with partial success and significant additional complexity. Systems such as H-Store \cite{Kallman+08} (and its commercial version, VoltDB) largely dispense with logging and rely instead on replication. These systems replace the logging bottleneck with a transaction dispatch bottleneck. Silo avoids the logging bottleneck by giving up the traditional total order in transactions. Avoiding total ordering is efficient but prevents the system from using any but the simplest of CC schemes---there is no practical way to implement repeatable read or snapshot isolation, for example, and transactions {\it cannot see their own writes without sacrificing performance}.

We advocate a sweet spot between the extremes of fully coordinated logging (multiple synchronization points per transaction) and fully uncoordinated logging (no synchronization at all). A transaction with a reasonably small write footprint can acquire a totally ordered commit timestamp, and reserve all needed space in the log, using a single global atomic memory operation. While technically a potential bottleneck, previous work has shown that such a system can scale to a few {\em millions} of commits per second \cite{TuZKLM13}, while still preserving ordering information that enables advanced concurrency control schemes. By way of comparison, the current world record in TPC-E is not quite 9ktps. We present this log manager in \secref{design}.

\vspace{2mm} 
{\bf Concurrency control.} 
Broadly speaking, there are two camps of CC methods: the pessimistic, e.g. two-phase locking (2PL), and the optimistic (OCC). Past theory work \cite{AgrawalCL87} has shown that pessimistic methods are superior to optimistic ones under high contention. In practice, this result requires that pessimistic methods can be implemented with sufficiently low overhead relative to their optimistic counterparts, which is not easy to achieve in practice. For example, a study of the SHORE storage manager reports roughly 25\% overhead for locking-based pessimistic methods \cite{HarizopoulosAMS08}; roughly speaking, that means that a lightweight OCC scheme would likely outperform it until its abort rate rises past 25\%.

Having said that, typical memory-optimized engines that employ lightweight OCC and running on modern commodity servers, leave some room for exploration. First, a contentious workload can easily drive abort rates to 25\% or higher, opening the door for heavier but more robust schemes. Second, modern systems achieve extraordinarily high throughput for short-running, partition-able transactions with predictable footprints, and it might be worth trading 25\% of peak performance for more robust behavior across a wider spectrum of transactional and mixed workloads. In other words, it is easier to tolerate losing 15-20\% of peak performance if net throughput is still several hundreds of thousands of transactions per second.


There are different flavor of optimistic, or opportunistic, CC. Many recent systems adopt a lightweight read validation step at the end of the transaction, during pre-commit. Read validation is very opportunistic, and somewhat brittle, leaving the system vulnerable to workloads where writers are likely to overwrite readers (causing them to abort).
Further, these optimistic schemes make no guarantee that it is safe to retry the transaction immediately after pre-commit fails, an issue that performance evaluations typically downplay by not retrying failed transactions at all. Ideally, a good CC scheme would provide a ``safe restart'' property \cite{PortsG12} where the system guarantees that a transaction will not fail twice because of the same conflict.
Regardless optimistic or pessimistic, the CC mechanism should not only have a low false positive rate detecting conflicts, but it should allow the system to detect doomed transactions as early as possible to minimize the amount of wasted work.

At least to our knowledge, there is no (publicly available) system that is both fast enough and has the appropriate internal infrastructure to support the implementation of robust CC schemes. The internal infrastructure matters terribly. It decides whether it is even possible to implement a particular CC scheme, and also which implementable schemes can be made practical. For example, the effort to enhance Postgres with serializable snapshot isolation (SSI) required a very large implementation effort, since the team had to integrate what it is essentially a lock manager with a purely multi-versioned system.\footnote{Once all this groundwork was done, extending Postgres to other CCs is relatively easy.} Even then, the achieved performance borders on unusable, due to severe bottlenecks in multiple parts of the system (including a globally serialized pre-commit phase, latch contention in the new lock manager, and existing scalability problems in the log manager).  Many of the design decisions we outline in \secref{design} were specifically taken in order to allow more flexibility in implementing multiple CC schemes without facing such severe performance trade-offs.

\vspace{2mm}
{\bf Physical layer.} 
Transaction processing systems typically depend on their low-level storage manager component to mediate thread interactions at the physical level.
The implementation of the storage manager is extremely tightly coupled to the CC scheme, making it difficult to modify or extend the CC schemes of legacy systems.  

One promising technique that provides desirable properties for both CC and physical contention is the notion of an {\em indirection array} \cite{SadoghiRCB13,Diaconu+13} for mapping logical object IDs to physical locations.
For example, an indirection array reduces the amount of logging required for updates in append-only systems, because updating a record only requires installing a new physical pointer at the appropriate logical array entry; without the indirection, creating a new version requires updating every reference to that record. Updating leaf entries of tree-based secondary indexes with large nodes can be especially expensive, as all nodes on the path from root to leaf must be rewritten. In addition to the additional I/O burden, frequent root node updates lead to severe physical contention and---depending on the CC implementation---false positive CC aborts. Write-optimized data structures such as LSM trees \cite{NeilCGO96,SearsR12} alleviate the I/O bottleneck but shift significant burden to readers, who must now reconcile multiple datasets when performing range scans. 

As a comparison, Silo does not employ indirection arrays. Instead, it performs in place updates: under normal circumstances the system maintains only a single committed version of an object, in addition to some number of uncommitted private copies (only one of which can be installed). In order to support large read-only transactions, a heavyweight copy-on-write snapshot mechanism must be invoked. These snapshots are too expensive to use with small transactions, and unusable by transactions that perform any writes. Similarly, Hekaton is technically multi-versioned, but the snapshot-plus-read-validation CC scheme it uses means that older versions become unusable to update transactions as soon as any overwrite commits. For all practical purposes, both systems are multi-versioned only for read-only transactions.

Indirection arrays are suitable for the physical implementation of CC for multi-versioned systems, as a single compare-and-swap (CAS) operation suffices to install a new version of an object. Similarly, presence of an uncommitted version makes write-write conflicts easy to detect and manage. Indirection also eases the reclamation of obsolete versions in the background, avoiding interference with foreground transaction processing.

With indirection arrays, even anti-caching~\cite{DeBrabantPTSZ13} is largely simplified: in the vanilla anti-caching algorithm all the secondary indexes have to be updated whenever a record is evicted from memory, with some considerable overhead. The indirection array gives a convenient place to replace an in-memory pointer with an on-disk pointer, and so in some sense acts as a lightweight buffer pool. 

There are also low-level reasons, detailed in \secref{design:oid}, for using indirection arrays. For example, space management becomes easier, and they admit a convenient analog to cache-friendly compact index structures such as CSB+Trees \cite{RaoR00}. These benefits do not justify adopting indirection, but they become convenient to use once the indirection layer is in place for other reasons.

\vspace{2mm}
{\bf Epoch-based resource management.}
Resource management is a key concern in any storage manager, and one of the most challenging aspects of resource management is ensuring that all threads in the system have a consistent view of the available resources, without imposing heavy per-access burdens. The infamous ABA problem from the lock-free literature is one example of what can go wrong if the system does not maintain invariants about the presence and status of in-flight accesses. Epoch-based resource management schemes such as RCU \cite{McKenneyS98} achieve this tracking at low cost by requiring only that readers inform the system whether they are {\em active} (possibly holding references to resources) or {\em quiescent} (definitely not holding any references to resources). These announcements can be made fairly infrequently, for example when a transaction commits or a thread goes idle. Resource reclamation then follows two phases: the system first makes the resource unreachable to new arrivals, but delays reclaiming it until all threads have quiesced at least once (thus guaranteeing that all thread-private references have died). Epoch-based resource management is especially powerful when combined with multi-versioning, as writers can coexist relatively peacefully with readers. Although epochs are traditionally fairly coarse-grained (e.g. hundreds of ms in Silo), we have implemented an epoch manager that is lightweight enough to use even at very fine time scales. As discussed in \secref{design:epochs}, ERMIA instantiates several epoch managers, all running at different time scales, to simplify all types of resource management in the system. 

%The flexibility in implementing various CC schemes is greatly enhanced if we can establish total ordering. One way to achieve total ordering, is through a centralized log, which can be used for establishing the transaction commit order. Therefore we opt using a centralized log, but we minimize the interaction with it to only a single communication per transaction. 
%In contrast, Silo employs a epoch-based ordering that is only partial. 

%% -*- tex-main-file:"rcu-cc.tex" -*-

\section{ERMIA: Fast and robust memory-optimized OLTP}
\seclabel{design}
In this section, we start with an overview of ERMIA, and then describe the key pieces of ERMIA, with a focus on why we choose the design trade-offs we do.

\subsection{Overview}
ERMIA is designed around epoch-based resource management and an extreme efficient, centralized log. \figref{ermia-arch} shows the major components in ERMIA and how they interact with each other. The log manager, TID manager, and garbage collector are three major entities that uses the epoch-based resource management machinery. To avoid log buffer contention~\cite{WangJ14}, each transaction acquires its log private log buffer (hence its global start stamp) from the log manager. Upon commit, the transaction will fix its global order in the log via a single CAS instruction and flush all its log records in one go (see~\ref{subsec:logging} for details). Although both log buffers and TIDs are managed by the epoch-based resource management mechanism, they use separate epoch lengths and can track stragglers efficiently. Transaction threads access the database through indexes. Different from traditional tree structures which give access to data in the leaf level, in ERMIA we embed in each index an indirection array, which can provide easy implementation of snapshot isolation and garbage collection. Indexed by object IDs, each array entry points to a chain of historic version for the object (tuple). The garbage collector periodically goes over all version chains and remove unnecessary, superseded versions that are not needed by any transactions.
\labeledfigure{ermia-arch}{Architecture of ERMIA.}

\subsection{Logging}
\label{subsec:logging}
The log manager is a pivotal component in most database engines. It provides---if anything does---a centralized point of coordination that other pieces of the system build off of and depend on. ERMIA's log manager generates a commit log sequence number (LSN) for the transaction and reserves log buffer space for the transaction's log records with a \textit{single global atomic operation}. Achieving this required two key insights: first, the transaction can combine its log records into large blocks, avoiding the redundancy of writing individual log record headers and reducing the number of trips to the log. Second, the LSN space need not be contiguous as long as we can still convert easily between an LSN and the corresponding disk address.

The first property arises from our use of append-only storage. We achieve the second property by assigning each LSN to a ``segment'' and storing its segment number in the low order bits; the LSN's position on disk can be determined by looking up, and subtracting off, its segment's starting offset (read-only). Transactions race to ``open'' a new segment if they obtain an LSN past the end of the current one. Unlucky transactions holding an LSN in the gap between two segments can simply discard it and request a new one. Segments can be as large as 100GB or more, however, so overflows will be rare. Once an LSN and segment have been assigned, the transaction verifies availability of space in the log's circular memory buffer (again, read-only); only in case the buffer is full will transactions have to block pending space, but disk arrays can readily absorb the sequential write-only I/O stream.

An additional feature of the log is that transactions acquire a commit LSN before entering pre-commit, allowing validation of multiple transactions to proceed smoothly in parallel; depending on the outcome of pre-commit, a transaction either writes its log entries or a skip record (to abort) to the reserved log block.

Finally, because the shared counter is implemented as a wait-free linked list, the transaction can notify the log writer that its block is ready to be flushed by simply flagging its node as ``dead'' (a blind store). The log writer periodically scans the list and writes out all log blocks that precede the oldest ``live'' buffer allocation; transactions do not touch each others' nodes, and the log writer only reads them and flags dead nodes for the garbage collector.

\subsection{Epoch-based resource management}
\kk{why RCU? what is it good for? - classifies data into different time windows, cleanup is done by background without interreing with foreground operations. a building piece to achieve lock-free resource mgmt. } We have developed a lightweight epoch management system that can track multiple timelines of differing granularities in parallel. A multi-transaction-scale epoch manager implements garbage collection of dead versions and deleted records, a medium-scale epoch manager implements read-copy-update (RCU) that manages physical memory and data structure usages~\cite{McKenneyS98}, and a very short timescale epoch manager tracks transaction IDs (TIDs), which we recycle aggressively (see details in \secref{tm}).

The key to efficiency here is to avoid flagging stragglers unless it is absolutely necessary (because coordinating with non-responsive threads is very expensive). Therefore, ERMIA does not attempt to reclaim resources for epoch $N$ until epoch $N+2$ begins. This way, potential stragglers have all of epoch $N+1$ to quiesce without penalty; however, epoch $N+3$ cannot begin until the last straggler from epoch $N$ completes. This four-phase scheme communicates far less with stragglers than the traditional two-phase \tianzheng{three-phase?} approach while maintaining the same worst-case timing bound. It allows us to track epochs at a very fine granularity when necessary. 

\subsection{Transaction management}
\seclabel{tm}
Each transactions in the system is assigned a slot in a global transaction state table when it begins. This fixed-size table holds the transaction's begin time (which is the log's end LSN at the time it started), status, and end time (if applicable). TIDs are a combination of table offset and epoch, with an epoch manager to prevent entries from being recycled too soon. Update transactions write their TIDs into each version they create, change their status to pre-commit, acquire a commit LSN (or are given one by an impatient peer), and finally commit atomically by changing their status to ``committed.'' A post-commit cleanup step involves replacing the transaction's TID with its commit LSN, at which point the state table entry is no longer needed and can be recycled by the epoch manager. Other transactions that encounter a TID in a version can reliably verify its commit status and age by visiting the transaction state table, and---if necessary---will help a peer enter pre-commit by acquiring a log block on its behalf. 

\subsection{Indirection arrays}
\seclabel{design:oid}
\kk{citation to mo sadoghi and hekaton?}
The indirection arrays used in ERMIA are very similar to the ones proposed in the literature. All logical objects are identified by an object ID (OID) that maps to a slot in an OID array that contains the physical pointer to data. The pointer may reference disk, or a chain of versions stored in memory. As with Hekaton, uncommitted versions are never written to disk; but unlike Hekaton, we dispense with delta records (too expensive to apply) and use pure copy-on-write. New versions can be installed by an atomic compare-and-swap operation, and an uncommitted record at the head of the chain constitutes a write lock for CC schemes that care to track write-write conflicts (as most do). 

\subsection{Concurrency control}
\seclabel{design:cc}

ERMIA has been designed from the ground up to allow efficient implementations of a variety of CC mechanisms, including Silo/Hekaton flavored read-set validation and snapshot isolation. Moreover, it allows efficient implementations of non-trivial CC schemes (other than simplistic read-set validation) to handle heterogeneous workloads gracefully and make them serializable (e.g., complex financial transactions represented by TPC-E). The components in ERMIA work together to make this possible: indirection arrays allow cheap (almost free) multi-versioning; at an extremely low overhead, the log gives total commit ordering, which is the key to implement snapshot isolation; the transaction manager can help determine a version's age easily.

Depending on the targeted workload, ERMIA can use read set validation, two-phase commit, snapshot isolation, or even serializable snapshot isolation (SSI)~\cite{Cahill08RF}. We focus on handling \textit{read-mostly} workloads gracefully while maintaining comparable performance for short, update-intensive workloads. Although not serializable, snapshot isolation is an ideal choice to start from: long, read-mostly transactions will have much higher chance to survive when compared to read-set validation. ERMIA uses SSI to guarantee serializability. In particular, our variant of SSI takes advantage of ERMIA's design to avoid sacrificing much performance as virtually all existing SSI based systems do. Below we briefly highlight our SSI variant for ERMIA.

\labeledfigure{ssi-dang-struct}{The ``dangerous structure'' that must exist in every serial dependency cycle under SI. A non-serializable schedule will have T1 and T2 read versions that are later overwritten by T2 and T3, respectively, with a T3 which overwrites a version that was read by T2 commits first.}

{\bf Stamp-based tracking.}
% basically copied from Ryan's SSI simulator.
SSI ensures serailizability by tracking the ``dangerous structure'' that must exist in every serial dependency cycle under SI as shown in \figref{ssi-dang-struct}~\cite{Cahill08RF}: T1 or T2 must abort if the read-write dependency exists and T3 committed first. Since ERMIA provides global ordering through the log, tracking and detecting the dangerous structure becomes straightforward. We associate each version with three stamps: \texttt{s0}, \texttt{s1}, and \texttt{s2}. \texttt{s0} indicates the versions creation timestamp, which is the commit timestamp of the transaction that created this version. \texttt{s2} records the smallest successor stamp (\texttt{s1}) among all the transactions' reads. An \texttt{rstamp} is also maintained in each version to indicate the commit stamp of the latest reader transaction. Whenever transaction T performs a read, it checks whether \texttt{s2} is set on any version. If so, T must abort (being the T1 of a dangerous structure where T3 committed first). T next checks whether \texttt{s1} is set. If so, T is the ``pivot'' and we prefer to abort it if T has overwritten any in-flight readers. If no such reader exists, then T is allowed to continue but it must remember the smallest \texttt{s1} it encounters, in order to re-check overwritten readers during pre-commit. T also remembers the smallest \texttt{s0} of any version it has read. At pre-commit, T checks again whether it is the ``pivot'' of any dangerous structure involving an in-flight T1, aborting if so. Otherwise, T can enter post-commit and finalize the three stamps in each version it created, with \texttt{s0} being its commit stamp, \texttt{s1} as the remembered \texttt{s0}, and \texttt{s2} as the remembered \texttt{s1}, if any was seen. There is no need to update in-flight readers because of the pivot test already performed. We next describe how the commit protocol works.

{\bf Commit protocol.}
We divide the commit process into two parts: pre-commit and post-commit. Pre-commit under SSI involves three major steps: (1) obtain a commit stamp; (2) examine the read set to find out the smallest \texttt{s1}, as shown in the first half of Algorithm~\ref{alg-ssi-commit}; (3) examine overwritten versions by iterating the write set to update T's rstamp. As shown in the second half of Algorithm~\ref{alg-ssi-commit}, during step 3, T must abort if we found there is any in-flight readers of any overwritten version or T's rstamp is greater than the minimal s1 we obtained in step 2 (i.e., T will be the T2 in a dangerous structure if committed). If the transaction survived pre-commit, version stamps will be updated. Note that these version stamps (except the commit stamp) are only needed at runtime; the log will store the version data and header without SSI stamps.

\begin{algorithm}
\begin{algorithmic}[1]
\STATE {\bf Input:} Committing transaction \textit{T}
\STATE T.cstamp = get\_commit\_stamp()
\STATE /* Get the smallest s1 from all reads. */
\STATE \texttt{ts1} = 0;
\FOR{$each~read~version~\texttt{v}:$}
\STATE T = overwriter transaction of \texttt{v}
\IF{$T_s~exists~and~T_s.cstamp~<~cstamp$}
\STATE Spin until T$_s$ is resolved
\IF{$T_s~is~committed$}
\STATE T.s1 = min(T$_s$.cstamp, T.s1)
\ENDIF
\ELSE
\STATE T.s1 = min(T$_s$.cstamp, v.s1)
\ENDIF
\ENDFOR
\STATE
\STATE /* Check writes. */
\FOR{$each~written~version~\texttt{w}:$}
\IF{$\texttt{w}~is~an~insert$}
\STATE \textbf{continue}
\ENDIF
\STATE \texttt{v} = the overwritten version
\FOR{$each~reader~transaction~T_r:$}
\IF{$T_r~is~active$}
\STATE T.abort()
\ENDIF
\IF{$T_r.cstamp~<~T.cstamp$}
\STATE continue
\ENDIF
\STATE T.rstamp = max(T.rstamp, w.rstamp)
\IF{$T.rstamp\geq~T.s1$}
\STATE T.abort()
\ENDIF
\ENDFOR
\ENDFOR
\end{algorithmic}
\label{alg-ssi-commit}
\caption{ERMIA SSI commit protocol.}
\end{algorithm}

{\bf Lightweight readers tracking.}
To track reads of a certain version when the overwriter's pre-commits, one might maintain a list of the TIDs of readers in each version. Readers will register themselves to the list when they read the version and deregister when they commit or abort. However, embedding a list in version headers will increase the size of a version significantly, causing more cache misses, especially for versions that are frequently read by multiple transactions (which is common in main-memory systems with massively parallel hardware). Since ERMIA executes each transaction from beginning to end by a single thread, without changing transaction context, we use a centralized TID list to record all in-flight transactions, and a bitmap in each old version to indicate the positions of readers in the centralized list. A transaction will register itself by setting the bitmap and putting its TID in the list's slot that corresponds to its thread ID. Note that the TID list is indexed by thread ID, so there is no contention upon transaction registration. The writer will be able to find out all readers through the bitmap and TID list at pre-commit.

{\bf Read optimization.}
Maintaining read sets is a major performance overhead of SSI and most CC schemes that guarantee serializability, especially for transactions with long reads. Existing optimizations usually focus on avoiding certain read-only transactions to participate in SSI~\cite{PortsG12}, which will not help reduce the overhead for \textit{read-mostly} transactions. In particular, most tuples in a database should be ``old'', i.e., not frequently updated. Long, read-mostly transactions will therefore have a higher chance of accessing these old tuples. Based on this observation, we optimize reads by guaranteeing any transaction that has read ``old'' versions can commit, unless it violates other conditions that necessitates its abort (e.g., a write-write conflict). In detail, a version's age is determined by subtracting its creation timestamp from the accessing transaction's begin timestamp. A version is old if its age is larger than a predefined threshold. Old versions will not be inserted to the read set. As a result, there is no way for the reader itself to go through SSI checks as usual during pre-commit. At pre-commit, if a transaction (the ``writer'') found itself overwrote an old version that was read by another transaction, it will have to abort, so that the readers can safely commit, avoiding forming the dangerous structure.

The above method could generate false positives: the TID slot indicated by the set bit in an old version's bitmap might already be re-allocated to another transaction when the writer enters pre-commit. For example, the reader could have already committed by the time the writer enters pre-commit. Such a schedule does not indicate a cycle in the dependency graph (so far), but the writer still has to abort. To reduce such false positives, we employ a \texttt{bstamp} field in each old version to record the latest reader's begin time stamp. The writer will only abort if the reader found in the centralized TID list has a begin timestamp older than the version's \texttt{bstamp}, eliminating most false positives caused by TID slot reuse.

{\bf Phantom protection.} ERMIA is amenable to various phantom protection mechanisms, such as hierarchical and key-range locking~\cite{KimuraGK12,Lomet93}. Since ERMIA's our current implementation is based on Silo~\cite{TuZKLM13}, we opt for the same tree-version validation mechanism---a lightweight, conservative approach---for fair comparison in the evaluation. The basic idea is to track and verify tree node versions, taking advantage of the fact that insertion in Masstree will version number change in affected leaf nodes. In addition to maintaining the read set, same as Silo, ERMIA also maintains a \textit{node set}, which maps from leaf nodes that fall into the range query to node versions. The node set is examined after read/write set validation. If any node's version does not match with the latest, the transaction will abort. Under SI, the verification can be done right before post-commit, while for SSI we verify the node set as the last step of pre-commit. Interested readers may refer to~\cite{TuZKLM13} for more details.

\subsection{Recovery}
\seclabel{design:recovery}

Recovery in ERMIA is straightforward because the log contains only committed work. OID arrays are the only real source of complexity, as they are volatile in-memory data structures that make it possible to find all other objects in the system. Logical objects (records) are physically logged, while physical data (allocator state and OID array contents) use logical logging. OID arrays are themselves objects stored in a master OID array, but they are updated in place to avoid overloading the log, with changes replayed by a log analysis step that reads only log block headers. This analysis step is very fast, because the skipped-over log payloads account for 90\% or more of the total log. In order to support efficient recovery, system transactions occasionally checkpoint the OID arrays using a fuzzy checkpointing mechanism to minimize the impact on user transactions. Because the log is the database, recovery only needs to rebuild the OID arrays in memory; anti-caching will take care of loading the actual data, though background pre-loading is highly recommended to minimize cold start effects.

%\subsection{Prototype Implementation}
%\seclabel{design:prototype}
%
%We implement a prototype of the ERMIA architecture and measure its performance.  For the implementation of the prototype we use a large fraction of the publicly available Silo codebase~\footnote{Silo's codebase can be downloaded from: https://github.com/stephentu/silo.}.  Silo uses the Masstree \cite{MaoKM12} as a cache-efficient index structure. 

%% -*- tex-main-file:"rcu-cc.tex" -*-

\section{Evaluation}
\seclabel{eval}
In this section, we compare the performance of ERMIA with SILO, a representative of the lightweight OCC camp, from two perspectives: 1) CC impact on performance and 2) scalability of underlying storage manager. 
The purposes of the performance evaluation are the followings. First, we want to show that the recent proposals for main-memory OLTP solutions with lightweight optimistic concurrency control schemes can perform well and are suitable for only on a limited subset of transactional applications, primarily due to limitations imposed by their (tightly-integrated) concurrency control mechanism. Second, the proposed ERMIA design offers a more efficient concurrency control, which allows its performance to remain high under contentious workloads. At the same time, ERMIA does not suffer significant sacrifices in performance, even in workloads where the specialized solutions shine. Lastly, we measure performance overhead of two CC schemes, SI and SSI, quantitatively. In addition to the CC impact on performance, we also show that ERMIA achieves multicore scalability thanks to scalable storage manager, mainly achieved by log manager, epoch resource manager and latch-free indirection array. 


\subsection{Experimental setup}
\seclabel{eval:setup}
% System
We used a 4-socket server with 6-core Intel Xeon E7-4807 processors, for a total of 24 physical cores and 48 hyperthreads. The machine has 64GB of RAM. We gave pre-faulted memory pool to both systems to avoid excessive page faults during runtime. The maximum number of worker threads are same with the number of physical cores in the machine, because it is enough to saturate CPUs in main-memory databases and the more workers contend for CPU, interferring each other. All worker threads were pinned to a dedicated core and NUMA node to minimize context switch penalty and inter-socket communication costs. Log records are written to tmpfs asynchronously and we used TCmalloc for memory allocation.

\subsection{TPC-E}
\seclabel{eval:tpce}
We first explore how ERMIA and SILO react to contention in the TPC-E which is a new standard OLTP benchmark. Although TPC-C has been dominantly used to evaluate OLTP systems performance for the past few decades, TPC-E was designed as a more realistic OLTP benchmark with modern features. TPC-E models brokerage firm activities and has more sophisticated schema model and transaction execution control. It is also known for higher read to write ratio than TPC-C. 

%TPCE + contention
Since TPC-E is not a heterogeneous workload, we introduce a new read-mostly analytic transaction that evaluates total assets of a customer accounts group and inserts the analytic activity log into \textit{analytic\_history} table. Total assets of the accounts are computed by joining \textit{holding\_summary} and \textit{last\_trade} tables. The vast majority of contentions occur between the analytic transaction and \textit{trade-result} transaction on \textit{customer\_account}, \textit{holding\_summary} and \textit{last\_trade} tables. 
The parameters we set for this experiment are 5000 customers, 500 scale factor and 10 initial trading days(the initial trading day parameter was limited by our machine's memory capacity). %We also replicate TPC-E's market emulator(MEE) and make it thread-local structure to avoid MEE's internal locking overhead. 
The analytic transaction takes 10\% out of the original TPC-E transaction mix. % TODO. new workload mix, footnote?

\labeledfigurewide{fig-tpce-robustness}{Total throughput (left); analytic transaction throughput(right) of TPCE-hybrid, varying contention degree}

% chart - Commit rates / Abort rates, varying contention degree. 
\figref{fig-tpce-robustness}(left) shows TPS with 24 worker threads under varying contention degree when we run TPC-E with the analytic transaction(TPCE-hybrid). To vary contention degree, we change the size of a customer account group to be scanned by the analytic transaction from 5\% to 20\% . The X-axis indicates the size of the target customer account group. SILO is slightly faster than ERMIA-SI at the small(5\%) contention level. As we increase contention degree, we can observe that ERMIA-SI outperforms SILO by ??\% and ??\%, at 10\% and 20\% contention level respectively. ERMIA-SSI delivers ??\% higher throughput than SILO, even with the expensive serializability guarantee cost. \kk{ this observation and numbers will be changed after experiments with more frequent analytic}
The main cause for the result is that the contention in the workload imposed heavy pressure on the SILO's OCC protocol; OCC enforced transactions to abort even if single tuple of their read-set is invalidated by updaters. Meanwhile, ERMIA endured the contention by protecting the read-sets from updaters effectively with snapshot isolation and distributed the contentions across multiple versions. we now focus on the throughput of the analytic transactions to see how well both systems support analytics on transactional data set.
%ERMIA-SI produces commits more than ten times than SILO at 20\% scan range. %TODO. put SSI number also.
As shown in the \figref{fig-tpce-robustness}(right), the analytic transactions were shoot down massively even under small contention degree on SILO. It is obviously negative implication to run heterogenous workload with OCC. ERMIA provides balanced query/transaction performance, while SILO extremely favors updaters. Thus, we can conclude that the OCC is not suitable to support emerging operational analytic workload. 

% scalability 
\labeledfigurewide{fig-tpce-scalability}{Throughput when running TPCE-hybrid(left); TPCE-original(right)}
\labeledfigurewide{fig-tpcc-scalability}{Throughput when running TPCC-contention(left); TPCC-original(right)}
In \figref{fig-tpce-scalability}(left), we fix contention degree to 10\% scan range and see scalability trends, increasing the number of workers. Overwhelmed by contentions, SILO does not achieve scalability, even with its excellent scalability of the underlying physical layer. ERMIA benefits from its robust CC scheme and its storage manager did not forestall achieving linear scalability. This figure shows that CC scheme's capability to deal with contention dictates overall performance in logical level, no matter how the underlying physical layer is scalable, in heterogeneous workload. We also performed the same experiment in the original TPC-E, without the analytic transaction. In \figref{fig-tpce-scalability}(right), we see lack of contentions did not take down SILO and both scalable storage managers scale reasonably over 24 cores. ERMIA-SI delivered similar throughput to SILO and ERMIA-SSI delivered ??\% lower peak performance. 

\subsection{TPC-C} 
\seclabel{eval:tpcc}
We also run TPC-C where lightweight OCC camp shine. This experiment will be focusing on evaluating storage manager's scalability, as TPC-C impose little pressure on CC scheme. TPC-C is well-known for update-heavy workload and small transaction footprints. Also, it is well-partitionable workload; conflicts can be reduced even further by partitioning and single-threadinging on the partition, setting ideal environment for the lightweight OCC. The only source of contention under partitioned TPC-C is multi-partition transaction, however, its impact on CC is dampened by small transaction footprints, avoiding the majority of read-write conflicts. % cross-partition transaction ratio, partition lock, 

% TPCC-original
We measure throughput of both systems in original TPC-C(TPCC-original) , varying the number of worker threads. All worker threads are given home warehouse and does not change the home warehouse during runtime. Cross partition transactinos take ??\%. \figref{fig-tpcc-scalability}(right) shows that both systems scale reasonably over 24 cores. Compared to SILO, ERMIA falls behind SILO by 20\% and ??\%, with SI and SSI respectively, at the peak performance for the following reasons. First, SILO has more lightweight physical layer as it goes by single-version store. Second, as we mentioned earlier, CC did not affect overall performance due to the lack of contention. Consequently, the lack of contention turned ERMIA's effort toward robustness into marginal overhead; the indirection array is at the center of this trade-off between TPC-C peak performance and robustness. The indirection array maintains snapshots to support CC scheme to distribute contentions. Whenver ERMIA accesses to database tuples, it has to chase pointers in the indirection array to find a visible version and incurs more cache footprints subsequently. ERMIA-SSI has extra cost to guarantee serializability; it keeps track of transaction dependency information and lookup them to identify presence of cyclic dependency. As we argued, the overhead is a cost to pursue our key design goal. We believe that robust performance for wider range of applications has greater value at the expense of insignificant performance loss in the specific workload.

% TPCC-random WH
We now change workers-warehouse binding. We enforced worker threads to pick a partition randomly, following non-uniform distribution, during runtime. The purpose of this modification is to bring reasonable amount of contentions in TPC-C and to see how the previous results change. We can see in the \figref{fig-tpcc-scalability}(right), the performance gap gets smaller than TPCC-original as SILO is affected by the contention again. In TPCC-contention, the throughput of SILO decreased by approximately 30\%, while ERMIA delivered 15\% slower performance than in TPCC-original. 

\subsection{Performance study}
\seclabel{eval:perf-study}

profiling and performance analysis with breakdown(pie or bar chart will look intuitively)

\begin{itemize}
\item cost of SI - indirection array  -10\%
\item cost of SSI - -15\%
\item physical layer's scalability : 1) Index - the biggest overhead 2) log mgr sustaints 24 core
\end{itemize}

While traversing the version chain, ERMIA has ?\% larger cache footprint and subsequent more cache misses than SILO. 

SSI overhead - serial checking overhead, phantom overhead. needs comparison with other SSI techniques.  

Scalability is achieved with various techniques collectively. Cache optimization, NUMA-aware memory management, epoch resource manager and latch-free indirection array help to achieve scalability. 

Log manager and total ordering overhead - marginal
Log manager was able to sustain 400K commits/s. we don't need to give up total ordering for scalability.
TID allocation and epoch manager's overhead was also negligible. 


%% -*- tex-main-file:"rcu-cc.tex" -*-

\section{Related work}
\seclabel{related}

In terms of concurrency control, one of the most important studies has been \cite{AgrawalCL87}. This modeling study shows that if the overhead of pessimistic two-phase locking can be comparable to the overhead of optimistic methods then the pessimistic one is superior. The same study shows that it is beneficial to abort transactions that are going to abort as soon as possible. That is corroborated by other studies as well, e.g. \cite{PortsG12}. We follow the findings, trying to detect conflicts early.
 
The indirection map, which is central to ERMIA's design, is a well-known technique, for example presented in \cite{SadoghiRCB13}.

Many of the memory-optimized systems adopt lightweight optimistic concurrency control schemes that are suitable only for a small fraction of transactional workloads.
The designs can be categorized in three categories: non-partitioning- and partitioning-based systems and clustered solutions. 

Silo's  \cite{TuZKLM13} employs a light-weight optimistic concurrency control that performs validations at pre-commit. That, as we showed in \secref{eval}, performs well only in a limited set of workloads. 
Microsoft Hekaton \cite{Diaconu+13} employs similar multi-versioning CC \cite{LarsonBDFPZ11}. It is worth mentioning that Hekaton also uses a technique similar to the indirection map, which we also use. 

%% \ippo{Partitioning-based}
H-Store (and its commercial version, VoltDB) is a characteristic partitioning-based system \cite{Kallman+08}. H-Store physically partitions each database to as many instances as the number of available processors, and each processor executes each transaction in serial order without interruption.  
Problems raise when the system has to execute mutli-site transactions, transactions that touch data from two or more separate database instances. Lots of work has been put in the area, including low overhead concurrency control mechanisms \cite{JonesAM10}, but also partitioning advisors that help to co-locate data that are frequently accessed in the same transactions, thereby reducing the frequency of multi-site transactions, e.g. \cite{CurinoJZM10,PavloJZ11,TranNST14}.
Hyper \cite{KemperN11} follows H-Store's single-threaded execution principle.  To scale up to multi-cores they employ the hardware transactional memory capabilities of the latest generation of processors \cite{LeisKN14}. 

DORA \cite{PandisJHA10}  employs logical partitioning 
PLP \cite{PandisTJA11} extends the data-oriented execution principle, by employing physiological partitioning. Under PLP the logical partitioning is reflected at the root level of the B+tree indexes that now are essentially multi-rooted. PLP is
Both DORA and PLP use Shore-MT's codebase \cite{JohnsonPHAF09}, which is a scalable but disk-optimized storage manager with significantly bloated codebase. Hence, their performance lacks in comparison with the memory-optimized proposals. Additionally, even though only logical, there is a certain overhead in the performance due to the partitioning mechanism employed.  

In addition to the work on scaling up the performance of transaction processing systems in mutlicore and multisocket environment, there has been also lots of interest on scaling out. Those scale out systems, such as Google's Spanner \cite{Corbett+12}, RAMP \cite{BailisFHGS14} and Calvin \cite{ThomsonA10}, emphasize the weakness of the partitioning-based camp. Because the easy to partition workloads and databases would have already been partitioned of different physical nodes. Therefore whatever data are assigned to a single node it would be quite difficult to further partition.  Hence the need of scalable multicore and multisocket transaction processing system designs.

%% -*- tex-main-file:"rcu-cc.tex" -*-

\section{Conclusion}
\seclabel{conclusion}

In this paper we underlined the weaknesses of recent OCC-based transaction processing system proposals, and presented a novel system with more robust performance due to its concurrency control.
The point we want to make is that concurrency control is a fundamental component of any transaction processing system, and that it cannot be baked in after the fact to system. Instead, transaction processing system designers should think of concurrency control from the beginning as the decision about the concurrency control mechanism will dictate most of the design decision of all the other components of the systems. 


%%\input{ermia-acks}

%% IP: We should paste the .bbl here for the camera-ready
\bibliographystyle{abbrv}
%%\bibliographystyle{acm-custom}
%\setlength{\bibsep}{4pt}
{%\scriptsize % no need for big print here...
\bibliography{../common/biblio}
}

%\balancecolumns % GM June 2007
% That's all folks!
\end{document}
